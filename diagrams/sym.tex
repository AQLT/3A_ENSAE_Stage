\PassOptionsToPackage{unicode}{hyperref}
\PassOptionsToPackage{hyphens}{url}
\PassOptionsToPackage{dvipsnames,svgnames*,x11names*}{xcolor}
%
\documentclass[
  10pt,
  a4paper, fleqn, french]{article}
\usepackage{amsmath,amssymb}
\usepackage{lmodern}
\usepackage[]{microtype}
\usepackage{nccmath}

\usepackage[landscape, margin=0in]{geometry}

\usepackage{stmaryrd}
\usepackage{multicol}
\usepackage{graphicx}
\usepackage{animate, dsfont, here, xspace}
%\usepackage{tikz}       
\usepackage{tikz,pgfplots}
 \pgfplotsset{compat=1.17}
 \usepackage{lscape}%paysage
%\includepdf[fitpaper=true, pages=-]{img/pdg.pdf}



\DeclareMathOperator{\e}{e}
\renewcommand{\P}{\mathds{P}} %Apparement \P existe déjà ?
\newcommand\N{\mathds{N}}
\newcommand\R{\mathds{R}}
%\newcommand\C{\mathds{C}}
%\newcommand\Z{\mathds{Z}}


\newcommand\1{\mathds{1}}
\newcommand{\E}[2][]{{\mathds{E}}_{#1}
  \def\temp{#2}\ifx\temp\empty
  \else
    \left[#2\right]
  \fi
}
\newcommand{\V}[2][]{{\mathds{V}}_{#1}
  \def\temp{#2}\ifx\temp\empty
  \else
    \left[#2\right]
  \fi
}
\newcommand\ud{\,\mathrm{d}}

% blocks

\DeclareMathOperator*{\argmax}{argmax}
\DeclareMathOperator*{\argmin}{argmin}


%%%%%%%%%%%%%%%%%%%%%%%%%%%%%%%%%%%%%%%%%%%%%%%%%%%%%%%%%%%%%%%%%%%%%%
%%%%%%%%%%%%%%%%%%%%%%%%%%%%%%%%%%%%%%%%%%%%%%%%%%%%%%%%%%%%%%%%%%%%%%
%%%%%%%%%%%%%%%%%%%%%%%%%%%%%%%%%%%%%%%%%%%%%%%%%%%%%%%%%%%%%%%%%%%%%%
%                                                                    %
%                           Styles tikz                              %
%                                                                    %
%%%%%%%%%%%%%%%%%%%%%%%%%%%%%%%%%%%%%%%%%%%%%%%%%%%%%%%%%%%%%%%%%%%%%%
%%%%%%%%%%%%%%%%%%%%%%%%%%%%%%%%%%%%%%%%%%%%%%%%%%%%%%%%%%%%%%%%%%%%%%
%%%%%%%%%%%%%%%%%%%%%%%%%%%%%%%%%%%%%%%%%%%%%%%%%%%%%%%%%%%%%%%%%%%%%%

\tikzstyle{global}=[very thick, shape = rectangle, rounded corners, inner sep=10pt, inner ysep=10pt]
\tikzstyle{generalbox} = [global, draw=red!50, fill=orange!10]
\tikzstyle{generaltitle} = [above=-0.3cm,rounded corners, rectangle, draw=black,fill=red!20]

\tikzstyle{box} = [global, draw=blue!60!green, fill=blue!5, very thick]
\tikzstyle{boxtitle} = [above=-0.3cm,rounded corners, rectangle, draw=black,fill=blue!20]
\tikzstyle{edge} = [black!90, ->, rounded corners, line width = 1pt, >=latex]
\tikzstyle{edgeequiv} = [black!90, <->, rounded corners, max width = 3cm,double distance = 1pt, >=latex]
\tikzstyle{edgegeneral} = [red!90, ->, rounded corners, max width = 6cm, line width = 1.5pt, >=latex]


\usepackage{varwidth}
\tikzset{
    max width/.style args={#1}{
        execute at begin node={\begin{varwidth}{#1}},
        execute at end node={\end{varwidth}}
    }
}

\setlength{\mathindent}{0pt}
\begin{document}%
\setlength{\abovedisplayskip}{0pt}
\setlength{\belowdisplayskip}{0pt}
\setlength{\abovedisplayshortskip}{0pt}
\setlength{\belowdisplayshortskip}{0pt}

\begin{tikzpicture}


%%%%%%%%%%%%%%%%%%%%%%%%
%%%%% Theorie gen %%%%%
%%%%%%%%%%%%%%%%%%%%%%%%
\node at (3,0) [generalbox, max width=8cm] (theoriegen){
\[
\begin{cases}
I(\theta,q,y_t,u_t)=\E{(\Delta^{q}M_\theta y_t-u_t)^{2}} \\
J(\theta, f, \omega_1,\omega_2)=\int_{\omega_1}^{\omega_2} f\left[\phi_\theta(\omega), \varphi_\theta (\omega)\right] \ud \omega
\end{cases}
\]
\begin{align*}
\hat\theta\in\argmin &\sum\alpha_i I(\theta,q_i,y_t,u_t^{(i)})+\beta_iJ(\theta, f_i,\omega_{1}^{(i)}, \omega_{2}^{(i)})\\
&sc.\quad C\theta=a&
\end{align*}%
};
\node[generaltitle] at (theoriegen.north) (theriegentitle) {Théorie générale};

%%%%%%%%%%%%%%%%%%%%%%%%
%%%%% Gray Thompson%%%%%
%%%%%%%%%%%%%%%%%%%%%%%%
\node at (0,7.3) [box, max width=8cm] (graythompson){
\begin{gather*}
y_t=\underbrace{\sum_{j=0}^d\beta_jt^j+\xi_t}_{=\mu_t}+\varepsilon_t,
\quad 
\begin{cases}
	\varepsilon_t &\text{bruit blanc,  }\V{\varepsilon_t}=\sigma^2 \\
	\xi_t &\text{non corrélé à }\varepsilon_t 
\end{cases}\\
\text{Critères :}\begin{cases}
    F_{GT}(\theta)=\E{(\hat{\mu}_t-\mu_t)^2} \\
    S_{GT}(\theta)=\E{ (\Delta^{d+1}\hat{\mu}_{t})^{2}}
\end{cases}, \,\hat{\mu}_{t} = M_\theta y_t
\end{gather*}
\begin{align*}
\hat\theta\in&\argmin \alpha F_{GT}(\theta) + (1-\alpha)S_{GT}(\theta)\\
&s.c.\quad X'_d\theta=e_1
\end{align*}
};
\node[boxtitle] at (graythompson.north) (graythompsontitle) {Gray, Thompson (1996)};


%%%%%%%%%%%%%%%%%%%%%%%%%%
%%%%% Guggemos et al %%%%%
%%%%%%%%%%%%%%%%%%%%%%%%%%


\node at (0,-8) [box, max width=5.5cm] (guggemos){
\[
\text{Critères :}\begin{cases}
F_g(\theta)=\sum_{j=-h}^h\theta_j^2 \\
S_g(\theta)=\sum_{j=-h}^h(\Delta^q\theta_j)^2
\end{cases}
\]
\begin{align*}
\hat\theta\in&\argmin \alpha F_g(\theta) + (1-\alpha)S_g(\theta)\\
&s.c.\quad X'_d\theta=e_1
\end{align*}
};
\node[boxtitle] at (guggemos.north) (guggemostitle) {Guggemos \emph{et al} (2018)};


%%%%%%%%%%%%%%%%%%%%%%%%%%%%%
%%%%% Proietti et Luati %%%%%
%%%%%%%%%%%%%%%%%%%%%%%%%%%%%


\node at (20,-8) [box, max width=8cm] (proiettiluati){
\[y_t=\sum_{j=0}^d\beta_jt^j+\varepsilon_t,
\quad \varepsilon_t \text{ bruit blanc}
\]
$\kappa = (\kappa_{-h},\dots, \kappa_h)$ noyaux, $K=diag(\kappa_{-h},\dots,\kappa_{h})$
\begin{gather*}
(\hat \beta_0,\dots, \hat \beta_d)\in\argmin
(y-X_p\hat \beta)'K(y-X_d\hat\beta) \\
\implies \hat\theta=\hat \beta_0 = KX_d(X'_dKX_d)^{-1}e_{1}
\end{gather*}
};

\node[boxtitle] at (proiettiluati.north) (proiettiluatititle) {Proietti et Luati (2008)};


%%%%%%%%%%%%%%%%
%%%%% RKHS %%%%%
%%%%%%%%%%%%%%%%

\node at (20,8) [box, max width=8cm] (rkhs){
$f_0(t)$ noyau continu, $P_i$ polynômes orthonormaux de $\mathbb L^2(f_0)$ et $K_p(t)=\sum_{i=0}^{d-1}P_i(t)P_i(0)f_0(t)$.

\begin{ceqn}
\[
\hat\theta_i = \frac{K_d(i/b)}{\sum_{j=-h}^h K_d(j/b)}
\]
\end{ceqn}
};
\node[boxtitle] at (rkhs.north) (rkhstitle) {Dagum et Bianconcini (2008) --- RKHS};


%%%%%%%%%%%%%%%%%%
%%%%% Noyaux %%%%%
%%%%%%%%%%%%%%%%%%

\node[boxtitle] at (10, -9) (henderson) {Henderson};
\node[boxtitle] at (10, -7) (macaulay) {Filtre Macaulay};
\node[boxtitle] at (10, -6) (epanechnikoff) {Filtre Epanechnikoff};
\node[boxtitle, max width=3cm] at (10, -5) (nadarayawatson) {Nadaraya-Watson ou estimateur par noyaux
};
\node[boxtitle] at (10, -3) (loess) {LOESS%
};


%%%%%%%%%%%%%%%%%%
%%%%% Liens %%%%%%
%%%%%%%%%%%%%%%%%%

%% Gray Thompson - Guggemos
\draw[edgeequiv] (graythompson.220)  -- (guggemos.152) node[pos = 0.7, right]{
$\xi_t =0$

$q=d+1$};

% Luati - RKHS
\draw[edgeequiv] (proiettiluatititle.north)--(rkhs.south) node[pos = 0.7, left]{
$f_0=$ version continue de $\kappa_i$

$b=h+1$};

%% Gray Thompson -Luati
\draw[edge] (graythompson.280) --++(0,-.5) -|  (proiettiluati.30)  node[pos = 0.15, above, max width = 8cm]{
$\alpha=1\;;\;\V{(\xi_{t-h},\dots,\xi_{t+h})}=K^{-1}\;;\;\sigma^2=0$};

%% Gray Thompson - Guggemos
\draw[edgeequiv,  max width = 8cm] (guggemos.south)  -- ++(0,-.5) -| (proiettiluati.south) node[pos = 0.2, below]{
Théorème d'Henderson (1916)%
};

%%%%%%%%%%%%%%%%%%%
%%%%%% Theorie gen

%% gray thomson
\draw[edgegeneral] (theoriegen.150)--(graythompson.south) node[midway, right]{
$F_{GT}(\theta)=I(\theta,0,y_t,M_\theta y_t)$

$S_{GT}(\theta)=I(\theta,d+1,y_t,0)$
};

%% guggemos
\draw[edgegeneral] (theoriegen.208)--(guggemostitle.north) node[near start, right]{
$F_g(\theta)=I(\theta,0,y_t,\E{M_\theta y_t})$

$S_g(\theta)=I(\theta,q,y_t,\E{M_\theta y_t})$

$y_t=\sum_{j=0}^p\beta_jt^j+\varepsilon_t, \quad \varepsilon_{t}\overset{i.i.d}{\sim}\mathcal{N}(0,\sigma^{2})$
};

%% proietti
\draw[edgegeneral] (theoriegen.east)-|(proiettiluatititle.150) node[near end, left]{
$y_t=\underbrace{\sum_{j=0}^d\beta_jt^j}_{g_t}+\varepsilon_t$,

$\varepsilon=
\begin{pmatrix}
    \varepsilon_{t-h} \\ \vdots \\ \varepsilon_{t+h}
\end{pmatrix}\sim \mathcal N (0,K)$

Critère minimisation : 

$I(\theta, 0, y_t, g_t)$
};
%%%%%%%%%%%%
%%%%%% Noyaux

%%% From Guggemos
%% Henderson
\draw[edge] (guggemos.east)--++(0.2,0)|-(henderson.west) node[near end, below]{
$d=2,\, q=3,$

$\alpha = 0$
};

%% Macaulay
\draw[edge] (guggemos.east)--++(0.2,0)|-(macaulay.west) node[near end, below]{
$q=d+1,\,\alpha=1$
};

%% Macaulay
\draw[edge] (guggemos.east)--++(0.2,0)|-(epanechnikoff.west) node[near end, below]{
$q=2,\,\alpha=0$
};

%%% From Proietti
%% Henderson
\draw[edge, max width = 6cm] (proiettiluati.west)--++(-0.2,0)|-(henderson.east) node[pos=0.8]{
\begin{align*}
    \textstyle
    \kappa_{j}^H = & \textstyle 
      \left[1-\frac{j^2}{(h+1)^2}\right]
      \left[1-\frac{j^2}{(h+2)^2}\right]\\
     &  \textstyle
      \left[1-\frac{j^2}{(h+3)^2}\right]
\end{align*}
};

%% Macaulay
\draw[edge] (proiettiluati.west)--++(-0.2,0)|-(macaulay.east) node[near end, below]{
$\kappa_j^U=1$
};

%% epanechnikoff
\draw[edge] (proiettiluati.west)--++(-0.2,0)|-(epanechnikoff.east) node[near end, below]{
$\kappa_j^E=1-\left\lvert\frac{j}{h+1}\right\rvert^2$
};

%% Nadaraya-Watson
\draw[edge] (proiettiluati.west)--++(-0.2,0)|-(nadarayawatson.east) node[near end, below]{
$d=0$
};
%% LOESS
\draw[edge] (proiettiluati.west)--++(-0.2,0)|-(loess.east) node[near end, below]{
$\kappa_j^{TC}=\left(
  1-
  \left\lvert
  \frac j {h+1}
  \right\lvert^3
\right)^3$
};

\end{tikzpicture}

\end{document}