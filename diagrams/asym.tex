\PassOptionsToPackage{unicode}{hyperref}
\PassOptionsToPackage{hyphens}{url}
\PassOptionsToPackage{dvipsnames,svgnames*,x11names*}{xcolor}
%
\documentclass[
  10pt,
  a4paper, fleqn]{article}
\usepackage{amsmath,amssymb}
\usepackage{lmodern}
\usepackage[]{microtype}
\usepackage[]{nccmath}

\usepackage[landscape, margin=0in]{geometry}

\usepackage{stmaryrd}
\usepackage{multicol}
\usepackage{graphicx}
\usepackage{animate, dsfont, here, xspace}
%\usepackage{tikz}       
\usepackage{tikz,pgfplots}
 \pgfplotsset{compat=1.17}
 \usepackage{lscape}%paysage
%\includepdf[fitpaper=true, pages=-]{img/pdg.pdf}


\DeclareMathOperator{\e}{e}
\renewcommand{\P}{\mathds{P}} %Apparement \P existe déjà ?
\newcommand\N{\mathds{N}}
\newcommand\R{\mathds{R}}
%\newcommand\C{\mathds{C}}
%\newcommand\Z{\mathds{Z}}


\newcommand\1{\mathds{1}}
\newcommand{\E}[2][]{{\mathds{E}}_{#1}
  \def\temp{#2}\ifx\temp\empty
  \else
    \left[#2\right]
  \fi
}
\newcommand{\V}[2][]{{\mathds{V}}_{#1}
  \def\temp{#2}\ifx\temp\empty
  \else
    \left[#2\right]
  \fi
}
\newcommand\ud{\,\mathrm{d}}

% blocks

\DeclareMathOperator*{\argmax}{argmax}
\DeclareMathOperator*{\argmin}{argmin}


%%%%%%%%%%%%%%%%%%%%%%%%%%%%%%%%%%%%%%%%%%%%%%%%%%%%%%%%%%%%%%%%%%%%%%
%%%%%%%%%%%%%%%%%%%%%%%%%%%%%%%%%%%%%%%%%%%%%%%%%%%%%%%%%%%%%%%%%%%%%%
%%%%%%%%%%%%%%%%%%%%%%%%%%%%%%%%%%%%%%%%%%%%%%%%%%%%%%%%%%%%%%%%%%%%%%
%                                                                    %
%                           Styles tikz                              %
%                                                                    %
%%%%%%%%%%%%%%%%%%%%%%%%%%%%%%%%%%%%%%%%%%%%%%%%%%%%%%%%%%%%%%%%%%%%%%
%%%%%%%%%%%%%%%%%%%%%%%%%%%%%%%%%%%%%%%%%%%%%%%%%%%%%%%%%%%%%%%%%%%%%%
%%%%%%%%%%%%%%%%%%%%%%%%%%%%%%%%%%%%%%%%%%%%%%%%%%%%%%%%%%%%%%%%%%%%%%

\tikzstyle{global}=[very thick, shape = rectangle, rounded corners, inner sep=10pt, inner ysep=10pt]
\tikzstyle{generalbox} = [global, draw=red!50, fill=orange!10]
\tikzstyle{submodelbox} = [very thick, shape = rectangle, rounded corners, draw=blue!60!green, fill=blue!5,scale = 0.9]

\tikzstyle{generaltitle} = [above=-0.3cm,rounded corners, rectangle, draw=black,fill=red!20]

\tikzstyle{box} = [global, draw=blue!60!green, fill=blue!5]
\tikzstyle{boxtitle} = [above=-0.3cm,rounded corners, rectangle, draw=black,fill=blue!20]
\tikzstyle{boxtitle2} = [rounded corners, rectangle, draw=black,fill=blue!20]
\tikzstyle{multimodelbox} = [global, draw=blue!60!green, fill=blue!2, dashed]

\tikzstyle{edge} = [blue!60!green, -, rounded corners, >=latex,double distance = 1pt, text = black]
\tikzstyle{edgeequiv} = [black!90, <->, rounded corners, max width = 3cm,double distance = 1pt, >=latex]
\tikzstyle{edgegeneral} = [red!90, ->, rounded corners, max width = 6cm, line width = 1.5pt, >=latex]


\usepackage{varwidth}
\tikzset{
    max width/.style args={#1}{
        execute at begin node={\begin{varwidth}{#1}},
        execute at end node={\end{varwidth}}
    }
}


\setlength{\mathindent}{0pt}
\begin{document}%
\setlength{\abovedisplayskip}{0pt}
\setlength{\belowdisplayskip}{0pt}
\setlength{\abovedisplayshortskip}{0pt}
\setlength{\belowdisplayshortskip}{0pt}


\begin{tikzpicture}


%%%%%%%%%%%%%%%%%%%%%%%%
%%%%% Theorie gen %%%%%
%%%%%%%%%%%%%%%%%%%%%%%%

\node at (3,0) [generalbox, max width=8cm] (theoriegen){
\[
\begin{cases}
I(\theta,q,y_t,u_t)=\E{(\Delta^{q}M_\theta y_t-u_t)^{2}} \\
J(\theta, f, \omega_1,\omega_2)=\int_{\omega_1}^{\omega_2} f\left[\phi_\theta(\omega), \varphi_\theta (\omega)\right] \ud \omega
\end{cases}
\]
\begin{align*}
\hat\theta\in\argmin &\sum\alpha_i I(\theta,q_i,y_t,u_t^{(i)})+\beta_iJ(\theta, f_i,\omega_{1}^{(i)}, \omega_{2}^{(i)})\\
&sc.\quad C\theta=a&
\end{align*}%
};
\node[generaltitle] at (theoriegen.north) (theriegentitle) {Théorie générale};

%%%%%%%%%%%%%%%%%%%%%%%%
%%%%% Gray Thompson%%%%%
%%%%%%%%%%%%%%%%%%%%%%%%
\node (graythomsonA) at (-4,  9.7) {};
\node (graythomsonB) at (4.5,  5.3) {};
\node (graythomsonequiv) at (0,  5.3) {};
\draw[multimodelbox] (graythomsonA) rectangle (graythomsonB);
\node[boxtitle2, below right] at (graythomsonA) (graythompsontitle) {Gray, Thompson (1996)};
\node at (0,9) [box, below, max width=6cm] (graythompson){
    Minimisation des révisions au filtre symétrique sous contrainte
};
%\node[boxtitle] at (graythompson.north) (graythompsontitle) {Gray, Thompson (1996)};


\node at (-2.5,7) [submodelbox, max width=3cm, below] (graythompson1){
    Filtre asymétrique sans biais 
    
$\implies X'_{p}\hat\theta^{a}=e_1$
};
\node at (2,7) [submodelbox, max width=3cm, below] (graythompson2){
Filtre asymétrique de biais constant 

$\implies X'_{p-1}\hat\theta^{a}=e_1$
};

\draw [edge] (graythompson) -- (graythompson1) node[right, midway]{Contrainte};
\draw [edge] (graythompson) -- (graythompson2);

%%%%%%%%%%%%%%%%%%%%%%%%%%
%%%%% Guggemos et al %%%%%
%%%%%%%%%%%%%%%%%%%%%%%%%%


\node at (0,-9) [box, max width=8cm] (guggemos){
$F_g(\theta)=\sum_{j=-h}^h\theta_j^2$

$S_g(\theta)=\sum_{j=-h}^h(\Delta^q\theta_j)^2$

$T_g(\theta)=\int_{0}^{\omega_1}\phi_\theta^2(\omega) \sin^2(\varphi_a(\omega)) \ud \omega$
\begin{align*}
\hat\theta&\in\argmin \nu_1 F_g(\theta) + \nu_2 S_g(\theta)+ (1-\nu_1-\nu_2)T_g(\theta)\\
&s.c.\quad X'_p\theta=e_1
\end{align*}
};
\node[boxtitle] at (guggemos.north) (guggemostitle) {Guggemos \emph{et al} (2018)};


%%%%%%%%%%%%%%%%%%%%%%%%%%%%%
%%%%% Proietti et Luati %%%%%
%%%%%%%%%%%%%%%%%%%%%%%%%%%%%


\node (proiettiluatiA) at (8,-10.8) {};
\node (proiettiluatiB) at (24.8,-3.5) {};
\draw[multimodelbox] (proiettiluatiA) rectangle (proiettiluatiB);
\node[boxtitle2, below left] at (proiettiluatiB) (proiettiluatititle) {Proietti et Luati (2008)};

\node at (20,-9) [box, max width=8cm, inner ysep=2pt] (proiettiluati){
\begin{center}\centering \textbf{Modèle général}\end{center}\vspace{-0.2cm}
\[
y=U\gamma+Z\delta+\varepsilon,\quad \varepsilon\sim\mathcal N (0,D)
\text{ et } \begin{bmatrix} U & Z \end{bmatrix} \subset X
\]

Minimisation des révisions à $\theta^s$ sous contrainte :
\[
U'_p\theta^a=U\theta^s, \quad U=\begin{bmatrix} U_p \\ U_f \end{bmatrix} 
\]
};

%\node[boxtitle] at (proiettiluati.north) (proiettiluatititle) {Proietti et Luati (2008)};


%%%% DAF %%%%
\node at (11,-9.7) [submodelbox, max width=4.5cm, above] (daf){
\begin{center}\centering \textbf{DAF}\end{center}\vspace{-0.5cm}
méthode symétrique mais avec moins de données
};


%%%% CQ %%%%
\node at (11,-8.5) [submodelbox, max width=4.5cm, above] (cq){
  \begin{center}\centering \textbf{CQ}\end{center}\vspace{-0.5cm}
  \[
  y_t=\gamma_0+\gamma_1 t+\gamma_2t^2+\delta t^3 +\varepsilon_t
  \]
  $\varepsilon_t$ bruit blanc et $\theta^a$ préserve tendances quadratiques
};

%%%% QL %%%%
\node at (11,-6.8) [submodelbox, max width=4.5cm, above] (ql){
  \begin{center}\centering \textbf{QL}\end{center}\vspace{-0.5cm}
  \[
  y_t=\gamma_0+\gamma_1 t+\delta t^2 +\varepsilon_t,
  \]
  $\varepsilon_t$ bruit blanc et $\theta^a$ préserve tendances linéaires
};


%%%% LC %%%%
\node at (20,-6) [submodelbox, max width=4.5cm, above] (lc){
\begin{center}\centering \textbf{LC/Musgrave}\end{center}\vspace{-0.5cm}
\[
y_t=\gamma_0+\delta t +\varepsilon_t, 
\]
$\varepsilon_t$ bruit blanc et $\theta^a$ préserve constantes
};

%%%% Liens
\draw [edge, max width = 2cm] (proiettiluati) -- (daf) node[below, midway] {
\[
\begin{cases}
  D=K^{-1}\\ 
  U=X_3\\ 
  Z=0
\end{cases}
\]
};

\draw[edge, max width = 2cm] (proiettiluati) -- (cq) node[above right=-0.2cm,near end] {
\[
\begin{cases}
U=X_2\\ 
Z=x_3\\
D=\sigma^2I
\end{cases}
\]
};

\draw [edge, max width = 2cm] (proiettiluati.160) |- (ql) node[above, near end] {
\[
\begin{cases}
U=X_1\\ 
Z= x_2\\
D=\sigma^2I
\end{cases}
\]
};
\draw [edge, max width = 2cm] (proiettiluati) -- (lc) node[right, midway] {
\[
\begin{cases}
U=X_0\\ 
  Z=x_1\\
  D=\sigma^2I
\end{cases}
\]
};

%%%%%%%%%%%%%%%%
%%%%% RKHS %%%%%
%%%%%%%%%%%%%%%%


\node at (20,8) [box, max width=8cm] (rkhs){
$f_0(t)$ noyau continu, $P_i$ polynômes orthonormaux de $\mathbb L^2(f_0)$ et $K_p(t)=\sum_{i=0}^{p-1}P_i(t)P_i(0)f_0(t)$.
$$
\hat\theta_i = \frac{K_p(i/b)}{\sum_{j=-h}^q K_p(j/b)}
$$
};
\node[boxtitle] at (rkhs.north) (rkhstitle) {Dagum et Bianconcini (2008) --- RKHS};


%%%%%%%%%%%%%%%%
%%%%% Wildi %%%%%
%%%%%%%%%%%%%%%%


\node at (10,7.6) [box, max width=8cm] (trilemna){
\begin{align*}
A_w(\theta)&= 2\int_0^{\omega_1}\left(\rho_s(\omega)-\rho_\theta(\omega)\right)^{2}h(\omega)\ud\omega\\
T_w(\theta)&= 8\int_0^{\omega_1}\rho_s(\lambda)\rho_\theta(\lambda)\sin^{2}\left(\frac{\varphi_\theta(\omega)}{2}\right)h(\omega)\ud\omega\\
S_w(\theta)&= 2\int_{\omega_1}^\pi\left(\rho_s(\omega)^{2}-\rho_\theta(\omega)\right)^{2}h(\omega)\ud\omega%\\
%R_w&= 8\int_{\omega_1}^\pi\rho_s(\lambda)\rho_\theta(\lambda)\sin^{2}\left(\frac{\varphi_\theta(\omega)}{2}\right)h(\omega)\ud\omega
\end{align*}
\[
\min\nu_1A_w + \nu_2T_w+(1-\nu_1-\nu_2)S_w
\]
};
\node[boxtitle] at (trilemna.north) (trilemnatitle) {Wildi, McElroy (2019)};



%%%%%%%%%%%%%%%%%%
%%%%% Liens %%%%%%
%%%%%%%%%%%%%%%%%%

%%%%%%%%%%%%%%%%%%%
%%%%%% Theorie gen

%% gray thomson
\draw[edgegeneral] (theoriegen.150)--(graythomsonequiv) node[midway, right]{
$y_t=\underbrace{\sum_{j=0}^d\beta_jt^j+\xi_t}_{=\mu_t}+\varepsilon_t$
};

%% guggemos
\draw[edgegeneral, max width = 8cm] (theoriegen.208)--(guggemostitle.north) node[near start, right]{
$F_g(\theta)=I(\theta,0,y_t,\E{L_\theta y_t})$

$S_g(\theta)=I(\theta,q,y_t,\E{L_\theta y_t})$

$T_g(\theta)=J(\theta,f\colon(\rho,\varphi)\mapsto\rho^2\sin(\varphi)^2,0, \,\omega_1)$

$y_t=\sum_{j=0}^p\beta_jt^j+\varepsilon_t, \quad \varepsilon_{t}\overset{i.i.d}{\sim}\mathcal{N}(0,\sigma^{2})$
};

\draw[edgegeneral, max width = 8cm] (theoriegen.0)-|(trilemna.south) node[near end, left]{
$\begin{cases}
    f_1(\rho,\varphi, \omega)=2\left(\rho_s(\omega)-\rho\right)^{2}h(\omega) \\
    f_2(\rho,\varphi, \omega)=8\rho_s(\omega)\rho\sin^{2}\left(\frac{\varphi}{2}\right)h(\omega)
\end{cases}$
\begin{align*}
A_w(\theta)&= J(\theta,f_1,0,\omega_1)\\
T_w(\theta)&= J(\theta,f_2,0,\omega_1)\\
S_w(\theta)&= J(\theta,f_1,\omega_1,\pi)\\
R_w(\theta)&= J(\theta,f_2,\omega_1,\pi).
\end{align*}
};
\end{tikzpicture}


\end{document}